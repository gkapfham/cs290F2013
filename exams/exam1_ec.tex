\documentclass[12pt]{article}             
\textwidth = 6.5in
\textheight = 9.05in
\topmargin 0.0in
\oddsidemargin 0.0in
\evensidemargin 0.0in

% set it so that subsubsections have numbers and they
% are displayed in the TOC (maybe hard to read, might want to disable)

\setcounter{secnumdepth}{3}
\setcounter{tocdepth}{3}

% define widow protection 
        
\def\widow#1{\vskip #1\vbadness10000\penalty-200\vskip-#1}

% define a little section heading that doesn't go with any number

\def\littlesection#1{
\widow{2cm}
\vskip 0.5cm
\noindent{\bf #1}
\vskip 0.1cm
\noindent
}

% A paraphrase mode that makes it easy to see the stuff that shouldn't
% stay in for the final proposal

\newdimen\tmpdim
\long\def\paraphrase#1{{\parskip=0pt\hfil\break
\tmpdim=\hsize\advance\tmpdim by -15pt\noindent%
\hbox to \hsize
{\vrule\hskip 3pt\vrule\hfil\hbox to \tmpdim{\vbox{\hsize=\tmpdim
\def\par{\leavevmode\endgraf}
\obeyspaces \obeylines 
\let\par=\endgraf
\bf #1}}}}}

\renewcommand{\baselinestretch}{1.2}    % must go before the begin of doc
\newtheorem{principle}{Principle}
\newtheorem{definition}{Definition}
\newtheorem{define}{Definition}
% go with the way that CC sets the margins

\begin{document}

%\vspace*{3in}

\begin{center}

{\bf Extra Credit} \\ 
CS 290: Principles of Software Development \\
Examination 1 \\ \mbox{} \\
Pledge: \\

\vspace*{.4in}

{\bf Rules:} You must answer all questions correctly in order to
receive 1 bonus point.

\end{center}

\noindent
Name the student who:

\begin{enumerate}

\item Normally volunteers to yell out ``you're late''!

\item Once told Professor Kapfhammer that he needed to ``get out more
  often.''

\item Has recently performed difficult balancing acts with an IBM
  X series laptop.

\item Selected a desktop wallpaper of an animal that does not exist in
  the real world.

\item Spent some time in Germany and documented his trip with a
  digital camera.

\item Lived in the country of Australia.

\end{enumerate}

\end{document}
