\documentclass[12pt,epsf,psfig,graphics]{article}             
\textwidth = 6.5in
\textheight = 9.05in
\topmargin 0.0in
\oddsidemargin 0.0in
\evensidemargin 0.0in

% set it so that subsubsections have numbers and they
% are displayed in the TOC (maybe hard to read, might want to disable)

\usepackage[T1]{fontenc}
\usepackage{mathptmx}

%\usepackage{graphics}

\setcounter{secnumdepth}{3}
\setcounter{tocdepth}{3}

% define widow protection 
        
\def\widow#1{\vskip #1\vbadness10000\penalty-200\vskip-#1}

% define a little section heading that doesn't go with any number

\def\littlesection#1{
\widow{2cm}
\vskip 0.5cm
\noindent{\bf #1}
\vskip 0.1cm
\noindent
}

% A paraphrase mode that makes it easy to see the stuff that shouldn't
% stay in for the final proposal

\newdimen\tmpdim
\long\def\paraphrase#1{{\parskip=0pt\hfil\break
\tmpdim=\hsize\advance\tmpdim by -15pt\noindent%
\hbox to \hsize
{\vrule\hskip 3pt\vrule\hfil\hbox to \tmpdim{\vbox{\hsize=\tmpdim
\def\par{\leavevmode\endgraf}
\obeyspaces \obeylines 
\let\par=\endgraf
\bf #1}}}}}

\renewcommand{\baselinestretch}{1.2}    % must go before the begin of doc
\newtheorem{principle}{Principle}
\newtheorem{definition}{Definition}
\newtheorem{define}{Definition}
% go with the way that CC sets the margins

\usepackage{listings}

\usepackage{color}

\definecolor{javared}{rgb}{0.6,0,0} % for strings
\definecolor{javagreen}{rgb}{0.25,0.5,0.35} % comments
\definecolor{javapurple}{rgb}{0.5,0,0.35} % keywords
\definecolor{javadocblue}{rgb}{0.25,0.35,0.75} % javadoc

\begin{document}

\lstset{language=Java,
basicstyle=\ttfamily,
keywordstyle=\color{javapurple}\bfseries,
stringstyle=\color{javared},
commentstyle=\color{javagreen},
morecomment=[s][\color{javadocblue}]{/**}{*/},
%numbers=left,
numberstyle=\scriptsize\color{black},
stepnumber=1,
numbersep=7pt,
tabsize=4,
showspaces=false,
showstringspaces=false}

% handle widows appropriately
\def\widow#1{\vskip #1\vbadness10000\penalty-200\vskip-#1}

\begin{center}

CS290: Principles of Software Development \\
Examination One\\
%Saturday December 11, 2004 \\

\end{center}

\noindent
Answer the five questions that are listed on the following pages.  You must provide answers to these questions on a
separate sheet of paper.  Please develop responses that clearly express your ideas in the most succinct manner possible.
You are not permitted to complete this examination in conjunction with any of your classmates.  Furthermore, you cannot
consult any outside references during this examination.  If you have questions concerning the following problems, then
please visit my office during the examination period.  If you leave the classroom to take the exam, then you are
responsible for checking the white board for status updates.

%\mbox{} \newline
%\mbox{} \newline

\begin{enumerate}
  
\item ({\bf 10 Points}) In {\em The Mythical Man Month} Frederick
  Brooks identifies some of the fundamental challenges associated with
  the engineering of computer software.  Answer the following
  questions about the concepts developed by Brooks.

  \begin{enumerate}
          
  \item ({\bf 6 Points}) One of the most challenging tasks associated
    with implementing a software system is managing the software
    development team.  Brooks presents several graphs that contain
    ``People'' on the horizontal axis and ``Months'' on the vertical
    one.  What does a curve on this graph look like if $\ldots$

    \begin{enumerate}

      \item ({\bf 3 Points}) $\ldots$ a task is perfectly partitionable?
    
      \item ({\bf 3 Points}) $\ldots$ a task is not partitionable at all?

    \end{enumerate}

  \item ({\bf 4 Points}) Using his experience at IBM as a foundation
    for his ideas, Brooks describes the complete evolution of a
    programming systems product.  Using a diagram with four distinct
    quadrants, please describe the full transformation of a program
    into a programming systems product.  Please make sure that your
    diagram clearly explains the cost overheads associated with this
    transformation.

  \end{enumerate}
        
\newpage

\item ({\bf 10 Points}) Software process models are used as
  abstractions that can help to explain different approaches to
  software development. Respond to the following questions about how
  process models can guide the creation of software systems.

        \begin{enumerate}
          
        \item ({\bf 4 Points}) Discuss the fundamental phases that
          are common to most software lifecycles.  Provide a one or two
          sentence description of each activity.  Please include an
          example of one or two tools that can support each of these
          activities. If no tool support is available, then clearly
          state why this is the case!  Your response should include a
          properly labeled figure that includes the phase and its
          input and output.
          
        \item ({\bf 2 Points}) Explain the similarities and
          differences between the waterfall lifecycle model and the
          spiral model proposed by Boehm.

        \item ({\bf 2 Points}) You can also use either an incremental
          or the iterative approach to developing software.  What
          are the similarities and differences between these two
          models?

        \item ({\bf 2 Points}) A node in an activity graph stands for
          a milestone in a software development project.  What
          is the meaning behind the nodes and edges in an activity
          graph when they have the following two configurations?

          \begin{enumerate}

            \item Nodes $N_i$ and $N_j$ that are connected by a solid
              edge from $N_i$ to $N_j$

            \item Nodes $N_k$ and $N_l$ that are connected by a dashed
              edge from $N_k$ to $N_l$

          \end{enumerate}

        \end{enumerate}

\newpage

\item ({\bf 10 Points}) Testing is a part of the software lifecycle
  that can be used for many different purposes.  Provide a response 
  to each of the following questions about testing.

\begin{enumerate}       
  
% \item ({\bf 3 Points}) Pfleeger and Atlee identify different types of activities such as unit, integration, and system
% 	testing.  What is the motivation behind performing these tasks?
% 
%% \item ({\bf 2 Points}) Software testing is useful because of the
%%   benefits that you enumerated in response to the first part of this
%%   question.  Yet, testing has several different limitations.  Discuss
%%   two of the limitations commonly associated with software testing.

	\item ({\bf 3 Points}) There are many software development tools that we can use to support the development and
		testing of software.  Furnish a full explanation of these commands:

		\begin{enumerate}
			\item {\tt :Ant compile}
			\item {\tt :Ant test}
			\item {\tt :Ant coverage}
		\end{enumerate}

\item ({\bf 2 Points}) What are the similarities and differences
  between the implementation and testing of hardware and software?  Is
  it easier to test hardware or software?  Why?

\item ({\bf 5 Points}) Verification and validation are two activities that software engineers commonly perform during
	the construction of a software application.  In your response to this part of the question, please define both of
	these terms and furnish a concrete example of an associated activity.  Finally, you should compare and
	contrast these terms.
  
\end{enumerate}

\newpage

\item ({\bf 10 Points}) Managers often use a software process in order
  to make decisions about when to release an application.  For
  example, suppose that you are a manager and you are determining
  which features will be a part of the next release of your program.
  Moreover, assume that for each new requirement $R_j$ you already
  know $C_j$, the cost associated with implementing the requirement
  and $B_j$, the monetary benefit for a program that contains this
  feature.

  In order to determine which requirements will be part of the next
  release for your application, you must choose from the requirements
  $R = \{ R_1, \ldots, R_n \}$ and ensure that (i) the implementation
  tasks are completed at no more than the total fixed cost $C$ and
  (ii) you maximize the total monetary benefit that your company will
  see when they release the tool.  Given cost and benefit information
  for each $R_j$ and the fixed cost $C$, how will you determine which
  requirements are included in the next release of your program?
    
\newpage

\item ({\bf 10 Points}) Software implementation is the process of
  translating the software specification, design, and architecture
  into an executable representation.  Answer the following questions
  about the implementation and evaluation of computer software.

\begin{enumerate}

  % \item ({\bf 4 Points}) Implement a data generator that takes a list
  %   of numbers as input and then produces a list of lists as output.
  %   For instance, please consider the following outputs for the input
  %   list 1 2 3 4.  From this example, you can see that the data
  %   generator must return all of the lists that may be obtained by
  %   swapping certain items in the input list.  In this specific
  %   instance, an input list of size four produces an output list that
  %   contains six individual lists.  You may implement your data
  %   generator in any programming language.

  %   \begin{enumerate}

  %   \item 2 1 3 4
  %     
  %   \item 3 2 1 4

  %   \item 4 2 3 1

  %   \item 1 3 2 4
  %       
  %   \item 1 4 3 2

  %   \item 1 2 4 3

  % \end{enumerate}

  % \item ({\bf 3 Points}) Provide a test plan for the data generator
  %   that you implemented in Part 5(a) of this question.  Your plan
  %   should identify the inputs and anticipated output for each of the
  %   tests.  You should also clearly explain why you decided to include
  %   each of the test cases in the overall plan.

  \item ({\bf 2 Points}) Jeffrey Voas poses the question ``can clean
    pipes produce dirty water?'' in an article of the same title from
    IEEE Software.  What is the meaning behind the concepts of ``clean
    pipes'' and ``dirty water''?  In the context of software
    engineering, what is your response to the question that Jeffrey Voas raises?

\item ({\bf 4 Points}) Figure~\ref{kinetic} provides an incorrect implementation of the {\tt computeVelocity} method.
	What are the mistake(s) in this method?  How would you fix them?

\item ({\bf 2 Points}) Provide a test, written in JUnit format, that {\em cannot} find the original defect.  Your
	response to this question must explain why the test cannot identify the problem.

\item ({\bf 2 Points}) Provide a test, written in JUnit format, that {\em can} find the original defect.  Your
	response to this question must explain why the test can identify the problem.



\end{enumerate}

\begin{figure}[t]
\begin{minipage}{6in}
  \lstset{numbers=left}
  \begin{lstlisting}
package edu.allegheny.kinetic;

import java.lang.Math;

public class Kinetic {

	public static String computeVelocity(int kinetic, int mass) {

		int velocity_squared = 0;
		int velocity = 0;
		StringBuffer final_velocity = new StringBuffer();

		if (mass != 0) {
			velocity_squared = 3 * (kinetic / mass);
			velocity = (int) Math.sqrt(velocity_squared);
			final_velocity.append(velocity);
		}

		else {
			final_velocity.append("Undefined");
		}

		return final_velocity.toString();
	}

}
  \end{lstlisting}

\end{minipage}
\caption{The {\tt computeVelocity} Method of the {\tt Kinetic} class.}
\label{kinetic}
\end{figure}


\end{enumerate}

\end{document}
