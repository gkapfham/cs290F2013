\documentclass{article}             
\textwidth = 6.5in
\textheight = 9.05in
\topmargin 0.0in
\oddsidemargin 0.0in
\evensidemargin 0.0in

\usepackage{bookman}

% set it so that subsubsections have numbers and they
% are displayed in the TOC (maybe hard to read, might want to disable)

\setcounter{secnumdepth}{3}
\setcounter{tocdepth}{3}

% define widow protection 
        
\def\widow#1{\vskip #1\vbadness10000\penalty-200\vskip-#1}

% define a little section heading that doesn't go with any number

\def\littlesection#1{
\widow{2cm}
\vskip 0.5cm
\noindent{\bf #1}
\vskip 0.1cm
\noindent
}

% A paraphrase mode that makes it easy to see the stuff that shouldn't
% stay in for the final proposal

\newdimen\tmpdim
\long\def\paraphrase#1{{\parskip=0pt\hfil\break
\tmpdim=\hsize\advance\tmpdim by -15pt\noindent%
\hbox to \hsize
{\vrule\hskip 3pt\vrule\hfil\hbox to \tmpdim{\vbox{\hsize=\tmpdim
\def\par{\leavevmode\endgraf}
\obeyspaces \obeylines 
\let\par=\endgraf
\bf #1}}}}}

\renewcommand{\baselinestretch}{1.2}    % must go before the begin of doc

% go with the way that CC sets the margins

\usepackage{amsthm}
\usepackage{latexsym}


\theoremstyle{definition}
\newtheorem{definition}{Definition}

\begin{document}

% handle widows appropriately
\def\widow#1{\vskip #1\vbadness10000\penalty-200\vskip-#1}

\begin{center}

CS290: Principles of Software Development \\
Examination 2 \\
Monday, October 29, 2007 \\

\end{center}

\noindent
Answer the five questions that are listed below.  You must provide
answers to these questions on a separate sheet of paper.  Please
develop responses that clearly express your ideas in the most succinct
manner possible.  You are not permitted to complete this examination
in conjunction with any of your classmates.  Furthermore, you cannot
consult any outside references during this examination.  If you have
questions concerning the problems that are listed below, please visit
my office during the examination period.  If you leave the classroom
to take the exam, you are responsible for checking the white board for
status updates.

%\mbox{} \newline

\vspace*{.2in}

\begin{enumerate}
  
\item ({\bf 10 Points}) In {\em The Mythical Man Month} Frederick
  Brooks states that there is no {\em silver bullet} for the field of
  software engineering.  That is, he believes that there is no
  technical or managerial development that will offer an order of
  magnitude improvement in productivity. In order to support his
  argument, Brooks describes several concepts in light of the
  productivity equation provided in Equation~\ref{productivity}.

        \begin{equation} \label{productivity}
        \mbox{{\em Time of task}} = \sum_{i = 1}^{n} 
                \mbox{{\em Frequency}}_i \times \mbox{{\em Time}}_i
        \end{equation}

        \begin{enumerate}
          
        \item ({\bf 4 Points}) Explain the notions of {\em essence}
          and {\em accident} as Frederick Brooks defines them in the
          {\em Mythical Man Month}.  Give one or two examples of each
          concept.
          
        \item ({\bf 4 Points}) Using Equation~\ref{productivity},
          explain why Brooks asserts that there is no silver bullet to
          solve the problems that are associated with constructing
          software.
        
        \item ({\bf 2 Points}) Brooks states that software engineers
          have developed a number of potential silver bullets.
          Discuss two candidates for silver bullets and state whether
          or not you believe that they are indeed silver bullets.

        \end{enumerate}

%\mbox{} \newline

\newpage

\item ({\bf 10 Points}) Software process models are used as
  abstractions that can help to explain different approaches to
  software development.  Process models apply the same fundamental
  activities in different manners.

        \begin{enumerate}
          
        \item ({\bf 5 Points}) Discuss the fundamental ``phases'' that
          are common to most software processes.  Provide a one or two
          sentence description of each activity.  Please include an
          example of one or two tools that can support each of these
          activities. If no tool support is available, clearly state
          why this is the case!  Your response should include a
          properly labeled figure that includes the phase and its
          input and output.
          
%%         \item ({\bf 3 Points}) Clearly explain the halting problem and
%%           select one phase of the software lifecycle and discuss how
%%           the halting problem is relevant to one of the activities in
%%           this phase.  Your response should include a graphic to
%%           explain the halting problem.

        \item ({\bf 5 Points}) Explain the similarities and
          differences between the waterfall lifecycle model (i.e., the
          ``traditional'' model that is described by Hamlet and
          Maybee) and the spiral model proposed by Boehm.

        \end{enumerate}

\newpage

\item ({\bf 10 Points}) Software testing is an integral part of a
  software development methodology.  Answer the following questions
  about software testing activities.

  \begin{enumerate}

  \item ({\bf 2 Points}) What are the purpose(s) of software testing?
          
  \item ({\bf 3 Points}) One alternative to execution-based software
    testing is software inspection.  Clearly define the purpose of
    inspection.  Your response should also include an description of
    the roles that people fill during an inspection.

  \item ({\bf 3 Points}) Testing is often broken down into three
    different types of testing practices.  Define the terms black-box
    testing, white-box testing, and grey-box testing.  Furthermore,
    explain the similarities and differences between these three types
    of testing.
    
  \item ({\bf 2 Points}) Software engineers should aim to construct
    programs that have high testability.  What is the domain-to-range
    ratio (DRR) for a method?  Your response should include an example
    of a method that has a ``good'' DRR and another that has ``poor''
    DRR.  Why do you think that these methods exhibit good and bad DRR
    values?

  \end{enumerate}

\newpage
         
\item ({\bf 10 Points}) Managers often use a software process in order
  to make decisions about when to release a software application.  For
  example, suppose that you are a project manager and you are
  determining which features will be a part of the next release of
  your program.  Moreover, assume that for each new requirement $R_j$
  you already know $C_j$, the cost associated with implementing the
  requirement and $B_j$, the monetary benefit for a program that
  contains this feature.

  In order to determine which requirements will be part of the next
  release for your application, you must choose from the requirements
  $R = \{ R_1, \ldots, R_n \}$ and ensure that (i) the implementation
  tasks are completed at no more than the total fixed cost $C$ and
  (ii) you maximize the total monetary benefit that your company will
  see when they release the tool.  Given cost and benefit information
  for each $R_j$ and the fixed cost $C$, how will you determine which
  requirements are included in the next release of your program?
    
\newpage 

\item ({\bf 10 Points}) Software engineers must delicately balance
  their focus on both the product and the software process.  Answer
  the following questions about the controversy between process and
  product.

\begin{enumerate}

\item ({\bf 5 Points}) What are the five levels associated with the
  capability maturity model (CMM)?  Your response should provide the
  name of each level and then a brief description of the
  characteristics that describe an organization at that level.

\item ({\bf 2 Points}) What is a version control repository?  Why do
  we use this type of repository?  Your response should include a
  figure that explains the basic operation of a centralized version
  control system.

\item ({\bf 3 Points}) What is the category-partition method?  Please
  use one or two sentences to explain each step in this method.
 
\end{enumerate}






\end{enumerate}

\end{document}



