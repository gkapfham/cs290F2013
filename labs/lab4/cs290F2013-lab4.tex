% CS 580 style
% Typical usage (all UPPERCASE items are optional):
%       \input 580pre
%       \begin{document}
%       \MYTITLE{Title of document, e.g., Lab 1\\Due ...}
%       \MYHEADERS{short title}{other running head, e.g., due date}
%       \PURPOSE{Description of purpose}
%       \SUMMARY{Very short overview of assignment}
%       \DETAILS{Detailed description}
%         \SUBHEAD{if needed} ...
%         \SUBHEAD{if needed} ...
%          ...
%       \HANDIN{What to hand in and how}
%       \begin{checklist}
%       \item ...
%       \end{checklist}
% There is no need to include a "\documentstyle."
% However, there should be an "\end{document}."
%
%===========================================================
\documentclass[11pt,twoside,titlepage]{article}
%%NEED TO ADD epsf!!
\usepackage{threeparttop}
\usepackage{graphicx}
\usepackage{latexsym}
\usepackage{color}
\usepackage{listings}
\usepackage{fancyvrb}
%\usepackage{pgf,pgfarrows,pgfnodes,pgfautomata,pgfheaps,pgfshade}
\usepackage{tikz}
\usepackage[normalem]{ulem}
\tikzset{
    %Define standard arrow tip
%    >=stealth',
    %Define style for boxes
    oval/.style={
           rectangle,
           rounded corners,
           draw=black, very thick,
           text width=6.5em,
           minimum height=2em,
           text centered},
    % Define arrow style
    arr/.style={
           ->,
           thick,
           shorten <=2pt,
           shorten >=2pt,}
}
\usepackage[noend]{algorithmic}
\usepackage[noend]{algorithm}
\newcommand{\bfor}{{\bf for\ }}
\newcommand{\bthen}{{\bf then\ }}
\newcommand{\bwhile}{{\bf while\ }}
\newcommand{\btrue}{{\bf true\ }}
\newcommand{\bfalse}{{\bf false\ }}
\newcommand{\bto}{{\bf to\ }}
\newcommand{\bdo}{{\bf do\ }}
\newcommand{\bif}{{\bf if\ }}
\newcommand{\belse}{{\bf else\ }}
\newcommand{\band}{{\bf and\ }}
\newcommand{\breturn}{{\bf return\ }}
\newcommand{\mod}{{\rm mod}}
\renewcommand{\algorithmiccomment}[1]{$\rhd$ #1}
\newenvironment{checklist}{\par\noindent\hspace{-.25in}{\bf Checklist:}\renewcommand{\labelitemi}{$\Box$}%
\begin{itemize}}{\end{itemize}}
\pagestyle{threepartheadings}
\usepackage{url}
\usepackage{wrapfig}
% removing the standard hyperref to avoid the horrible boxes
%\usepackage{hyperref}
\usepackage[hidelinks]{hyperref}
% added in the dtklogos for the bibtex formatting
\usepackage{dtklogos}
%=========================
% One-inch margins everywhere
%=========================
\setlength{\topmargin}{0in}
\setlength{\textheight}{8.5in}
\setlength{\oddsidemargin}{0in}
\setlength{\evensidemargin}{0in}
\setlength{\textwidth}{6.5in}
%===============================
%===============================
% Macro for document title:
%===============================
\newcommand{\MYTITLE}[1]%
   {\begin{center}
     \begin{center}
     \bf
     CMPSC 290\\Principles of Software Development\\
     Fall 2013
     \medskip
     \end{center}
     \bf
     #1
     \end{center}
}
%================================
% Macro for headings:
%================================
\newcommand{\MYHEADERS}[2]%
   {\lhead{#1}
    \rhead{#2}
    %\immediate\write16{}
    %\immediate\write16{DATE OF HANDOUT?}
    %\read16 to \dateofhandout
    \def \dateofhandout {September 25, 2013}
    \lfoot{\sc Handed out on \dateofhandout}
    %\immediate\write16{}
    %\immediate\write16{HANDOUT NUMBER?}
    %\read16 to\handoutnum
    \def \handoutnum {5}
    \rfoot{Handout \handoutnum}
   }

%================================
% Macro for bold italic:
%================================
\newcommand{\bit}[1]{{\textit{\textbf{#1}}}}

%=========================
% Non-zero paragraph skips.
%=========================
\setlength{\parskip}{1ex}

%=========================
% Create various environments:
%=========================
\newcommand{\PURPOSE}{\par\noindent\hspace{-.25in}{\bf Purpose:\ }}
\newcommand{\SUMMARY}{\par\noindent\hspace{-.25in}{\bf Summary:\ }}
\newcommand{\DETAILS}{\par\noindent\hspace{-.25in}{\bf Details:\ }}
\newcommand{\HANDIN}{\par\noindent\hspace{-.25in}{\bf Hand in:\ }}
\newcommand{\SUBHEAD}[1]{\bigskip\par\noindent\hspace{-.1in}{\sc #1}\\}
%\newenvironment{CHECKLIST}{\begin{itemize}}{\end{itemize}}


\usepackage[compact]{titlesec}

\begin{document}
\MYTITLE{Laboratory Assignment Four: Team-Based Implementation and Testing of a Program}
\MYHEADERS{Laboratory Assignment Four}{Due: October 9, 2013}

\section*{Introduction}

In the previous laboratory assignments, you have been learning about the tools that we will use this semester to
specify, design, implement, test, and document Java programs. Moreover, our past class session have introduced you to
the key concepts associated with software engineering.  In this assignment, you and your team will follow the phases of
the software development life cycle and employ the concepts that we have studied in class  to implement and test a
program. Two weeks from now, you will present your solution and demonstrate your test suite and program.

\section*{Specifying the Requirements}

You are responsible for implementing a data generator that should take a list of numbers as input and produce a list of
lists as output.  You are given the following specification for the data generator.

\begin{quote}
For an input list of objects, denoted $L$, the data generator must produce all of the lists that can be obtained by swapping two
adjacent items in $L$.
\end{quote} 	

\noindent
For list $L = \{1, 2, 3, 4\}$, the customer wants the data generator to output $L=\{L_1, \ldots, L_6\}$ with 

\begin{itemize}
	\item[] $L_1 = \{2, 1, 3, 4\}$
	\item[] $L_2 = \{3, 2, 1, 4\}$
	\item[] $L_3 = \{4, 2, 3, 1\}$
	\item[] $L_3 = \{1, 3, 2, 4\}$
	\item[] $L_4 = \{1, 4, 3, 2\}$
	\item[] $L_6 = \{1, 2, 4, 3\}$
\end{itemize} 

The customer knows that the component that you must create will be a part of a larger system that has not yet been
fully implemented.  You are responsible for implementing this data generator so that it functions according to the
provided specification.  However, please note that the stated requirements may not be entirely correct!  It is the job
of your team to interact with the customer to ensure that the system is implemented as desired.  Using \LaTeX, you
should write a requirements document that fully explains the inputs, outputs, and behavior of the data generator.

\section*{Designing the System}

Working with the members of your team and leveraging the content in the requirements document, you should develop a
design for your system.  As you are finalizing the object-oriented design, you should try to develop answers to relevant
questions such as: How many classes will you use? What will be the relationship between the classes? What methods will
the classes have? What will be the inputs and outputs of the methods?  Is the design testable?  After answering these
questions, you should use \LaTeX\  to write a design document with text and diagrams that explain the system.

\section*{Implementing and Testing the Program}

Using the requirements and design document, your team must implement and test the data generator. You should focus on
implementing a program that is both correct and efficient. Just like in the previous laboratory assignment, your
implementation must include the following:

\begin{enumerate}
	\item A build system with rules for building, cleaning, testing, and running the program
	\item A high-coverage test suite that effectively tests all of the classes in the program
	\item A coverage report that was produced by the JaCoCo coverage monitoring tool
	\item Fully documented Java source code that completely implements the requirements
\end{enumerate}

Since you cannot exhaustively test this application, you must decide what types of inputs you will create in the test
cases.  You will also need to determine how you will know that the output of the data generator is correct.  For
instance, you should consider checking the following conditions:

\begin{enumerate}
	\item The contents of the output list only contain entities from the input list
	\item The output list has the correct number of sublists
	\item The output list contains all of the specified sublists
\end{enumerate}

\section*{Summary of the Required Deliverables}

This assignment invites your team to submit one printed version of the following files:

\begin{enumerate}
	
	\item A description of and justification for your team's chosen organization, roles, and tool support
	\item A document that clearly specifies the inputs, outputs, and behavior of the data generator
	\item A document that explains the data generator's design, with details about classes and methods
	\item All of the implementation artifacts (e.g., build system, source code, and the coverage report) 

\end{enumerate}

You must also ensure that the instructor has read access to your Bitbucket repository that is named according to the
convention {\tt cs290F2013-lab4-team{\em k}}, with {\tt {\em k}} representing the number of your assigned team.  Your
repository should contain all of the deliverables that you produced during the completion of this assignment.  Please
see the instructor if you have any questions.

\end{document}
