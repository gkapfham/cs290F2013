\input{labspre.tex}

\usepackage[compact]{titlesec}

\begin{document}
\MYTITLE{Laboratory Assignment Four: Team-Based Implementation and Testing of a Program}
\MYHEADERS{Laboratory Assignment Four}{Due: October 9, 2013}

\section*{Introduction}

In the previous laboratory assignments, you have been learning about the tools that we will use this semester to
specify, design, implement, test, and document Java programs. Moreover, our past class session have introduced you to
the key concepts associated with software engineering.  In this assignment, you and your team will follow the phases of
the software development life cycle and employ the concepts that we have studied in class  to implement and test a
program. Two weeks from now, you will present your solution and demonstrate your test suite and program.

\section*{Software Development Assignment}

You are responsible for implementing a data generator that should take a list of numbers as input and produce a list of
lists as output.  You are given the following specification for the data generator.

\begin{quote}
For an input list of numbers $L$, the data generator must produce all of the lists that can be obtained by swapping two
adjacent items in $L$.
\end{quote} 	

\section*{Summary of the Required Deliverables}

This assignment invites your team to submit one printed version of the following files:

\begin{enumerate}
	
	\item A description of the input(s), output(s), and behavior of the basic Eclim commands 
	\item A description of the output(s) and behavior of Vim's implementation and testing commands
	\item The first and final coverage reports produced by the JaCoCo coverage analysis tool
	\item A discussion of the defect that you found in the Kinetic Java class
	\item The final version of the Kinetic, KineticTest, and AllTests Java classes
	\item Screenshots demonstrating that your program and test suite run correctly
\end{enumerate}

You must also ensure that the instructor has read access to your Bitbucket repository that is named according to the
convention {\tt cs290F2013-lab3-team{\em k}}, with {\tt {\em k}} representing the number of your assigned team.  Please
ensure that your team does not modify the source code in the Kinetic repository --- instead, it is your responsibility
to make a new repository that will contain your team's version of Kinetic.  Each team should submit the appropriately
documented and tested source code for Kinetic --- you must turn in all of the Java source code that you created
or modified for this assignment. Please see the instructor if you have questions about these deliverables.

\end{document}
