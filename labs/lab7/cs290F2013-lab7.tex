\input{labspre.tex}

\usepackage[compact]{titlesec}

\begin{document}
\MYTITLE{Laboratory Assignment Seven: Using Mutation Analysis to Evaluate Test Suites}
\MYHEADERS{Laboratory Assignment Seven}{Due: November 13, 2013}

\section*{Introduction}

In the previous laboratory assignments, you have learned about the tools that we will use during the remainder of this
semester to specify, design, implement, test, and document Java programs.  You have also had several experiences with
working in progressively larger teams to complete the phases of the software life cycle, with a recent focus on the
elicitation of software requirements, the planning of software projects, and the analysis of   In this assignment, you
and your team will install, configure, and use software tools that automatically calculate design and implementation
metrics that characterize programs and test suites. One week from now, you will give a demonstration to the instructor
to show that you are able to perform mutation analysis with MAJOR.

\section*{Mutation Analysis with MAJOR}

\section*{Summary of the Required Deliverables}

This assignment invites your team to submit one printed version of the following files:
\vspace*{-.1in}
\begin{enumerate}
	\itemsep0em 
	\item A description of and justification for your team's chosen organization, roles, and tool support
	\item A document that clearly explains the meaning of JDepend's design quality metrics 
	\item An analysis of the values of design quality metrics for three previously implemented systems 
	\item A document that clearly explains the meaning of the metrics calculated by JavaNCSS
	\item An analysis of the values of the implementation metrics for the three systems that you chose
	\item Modified implementation artifacts of your chosen systems (e.g., build system and source code) 
	\item The slides of the presentation that you will give at the start of the next laboratory session
\end{enumerate}
\vspace*{-.1in}

You must also ensure that the instructor has read access to your Bitbucket repository that is named according to the
convention {\tt cs290F2013-lab6-team{\em k}}, with {\tt {\em k}} representing the number of your assigned team.  Your
repository should contain all of the deliverables that you produced during the completion of this assignment.  Please
see the instructor if you have any questions.

\end{document}
