\input{labspre.tex}

\usepackage[compact]{titlesec}

\begin{document}
\MYTITLE{Laboratory Assignment Seven: Using Mutation Analysis to Evaluate Test Suites}
\MYHEADERS{Laboratory Assignment Seven}{Due: November 13, 2013}

\section*{Introduction}

In the previous laboratory assignments, you learned about the tools that we will use to specify, design, implement,
test, and document Java programs.  You also had several experiences with working in progressively larger teams to
complete the phases of the software life cycle, with a recent focus on the elicitation of software requirements, the
planning of software projects, and the analysis of a project's design and implementation.  In this assignment, you will
install and use a mutation analysis tool that will help you to determine the quality of a test suite.  During the final
project, you can use this tool to better gauge how effectively you are testing your code. A week from now, you will
demonstrate to the instructor that you are able to perform mutation analysis.

\section*{Mutation Analysis with MAJOR}

Mutation analysis purposefully inserts hypothetical faults into the source code of your program and then checks to see
if the test cases can find these faults.  If the at least one test case fails after the insertion of the mutant, then
the tests can find this type of fault and we say that the mutant was killed.  However, it is also possible that the
tests will not be capable of finding the mutation fault. What does it mean if the mutant is alive after the execution of
the test suite?

MAJOR is a mutation analysis tool that integrates into the standard Java compiler. You can learn more about MAJOR by
visiting the Web site \url{http://iai.mathematik.uni-ulm.de/en/research/major.html}.  After visiting this Web site, you
should continue to learn about MAJOR by studying the provided tutorial and the papers referenced by the tutorial.  Once
you understand the basics of MAJOR, you should ask the instructor to give you read access to MAJOR's Git repository
hosted on Bitbucket. Once the you have access to the repository, you should type {\tt git clone
git@bitbucket.org:gkapfham/major-cs290f2013.git} in your terminal window.  Now, create your own Git repository that
follows the naming convention that we use for laboratory assignments and move all of MAJOR's code into it. 



\section*{Summary of the Required Deliverables}

This assignment invites your team to submit one printed version of the following files:
\vspace*{-.1in}
\begin{enumerate}
	\itemsep0em 
	\item A description of and justification for your team's chosen organization, roles, and tool support
	\item A listing of the steps that you will take during the demonstration of MAJOR
\end{enumerate}
\vspace*{-.1in}

You must also ensure that the instructor has read access to your Bitbucket repository that is named according to the
convention {\tt cs290F2013-lab7-team{\em k}}, with {\tt {\em k}} representing the number of your assigned team.  Your
repository should contain all of the deliverables that you produced during the completion of this assignment.  Please
see the instructor if you have any questions.

\end{document}
