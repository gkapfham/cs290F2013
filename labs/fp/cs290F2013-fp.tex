% CS 580 style
% Typical usage (all UPPERCASE items are optional):
%       \input 580pre
%       \begin{document}
%       \MYTITLE{Title of document, e.g., Lab 1\\Due ...}
%       \MYHEADERS{short title}{other running head, e.g., due date}
%       \PURPOSE{Description of purpose}
%       \SUMMARY{Very short overview of assignment}
%       \DETAILS{Detailed description}
%         \SUBHEAD{if needed} ...
%         \SUBHEAD{if needed} ...
%          ...
%       \HANDIN{What to hand in and how}
%       \begin{checklist}
%       \item ...
%       \end{checklist}
% There is no need to include a "\documentstyle."
% However, there should be an "\end{document}."
%
%===========================================================
\documentclass[11pt,twoside,titlepage]{article}
%%NEED TO ADD epsf!!
\usepackage{threeparttop}
\usepackage{graphicx}
\usepackage{latexsym}
\usepackage{color}
\usepackage{listings}
\usepackage{fancyvrb}
%\usepackage{pgf,pgfarrows,pgfnodes,pgfautomata,pgfheaps,pgfshade}
\usepackage{tikz}
\usepackage[normalem]{ulem}
\tikzset{
    %Define standard arrow tip
%    >=stealth',
    %Define style for boxes
    oval/.style={
           rectangle,
           rounded corners,
           draw=black, very thick,
           text width=6.5em,
           minimum height=2em,
           text centered},
    % Define arrow style
    arr/.style={
           ->,
           thick,
           shorten <=2pt,
           shorten >=2pt,}
}
\usepackage[noend]{algorithmic}
\usepackage[noend]{algorithm}
\newcommand{\bfor}{{\bf for\ }}
\newcommand{\bthen}{{\bf then\ }}
\newcommand{\bwhile}{{\bf while\ }}
\newcommand{\btrue}{{\bf true\ }}
\newcommand{\bfalse}{{\bf false\ }}
\newcommand{\bto}{{\bf to\ }}
\newcommand{\bdo}{{\bf do\ }}
\newcommand{\bif}{{\bf if\ }}
\newcommand{\belse}{{\bf else\ }}
\newcommand{\band}{{\bf and\ }}
\newcommand{\breturn}{{\bf return\ }}
\newcommand{\mod}{{\rm mod}}
\renewcommand{\algorithmiccomment}[1]{$\rhd$ #1}
\newenvironment{checklist}{\par\noindent\hspace{-.25in}{\bf Checklist:}\renewcommand{\labelitemi}{$\Box$}%
\begin{itemize}}{\end{itemize}}
\pagestyle{threepartheadings}
\usepackage{url}
\usepackage{wrapfig}
% removing the standard hyperref to avoid the horrible boxes
%\usepackage{hyperref}
\usepackage[hidelinks]{hyperref}
% added in the dtklogos for the bibtex formatting
\usepackage{dtklogos}
%=========================
% One-inch margins everywhere
%=========================
\setlength{\topmargin}{0in}
\setlength{\textheight}{8.5in}
\setlength{\oddsidemargin}{0in}
\setlength{\evensidemargin}{0in}
\setlength{\textwidth}{6.5in}
%===============================
%===============================
% Macro for document title:
%===============================
\newcommand{\MYTITLE}[1]%
   {\begin{center}
     \begin{center}
     \bf
     CMPSC 290\\Principles of Software Development\\
     Fall 2013
     \medskip
     \end{center}
     \bf
     #1
     \end{center}
}
%================================
% Macro for headings:
%================================
\newcommand{\MYHEADERS}[2]%
   {\lhead{#1}
    \rhead{#2}
    %\immediate\write16{}
    %\immediate\write16{DATE OF HANDOUT?}
    %\read16 to \dateofhandout
    \def \dateofhandout {September 25, 2013}
    \lfoot{\sc Handed out on \dateofhandout}
    %\immediate\write16{}
    %\immediate\write16{HANDOUT NUMBER?}
    %\read16 to\handoutnum
    \def \handoutnum {5}
    \rfoot{Handout \handoutnum}
   }

%================================
% Macro for bold italic:
%================================
\newcommand{\bit}[1]{{\textit{\textbf{#1}}}}

%=========================
% Non-zero paragraph skips.
%=========================
\setlength{\parskip}{1ex}

%=========================
% Create various environments:
%=========================
\newcommand{\PURPOSE}{\par\noindent\hspace{-.25in}{\bf Purpose:\ }}
\newcommand{\SUMMARY}{\par\noindent\hspace{-.25in}{\bf Summary:\ }}
\newcommand{\DETAILS}{\par\noindent\hspace{-.25in}{\bf Details:\ }}
\newcommand{\HANDIN}{\par\noindent\hspace{-.25in}{\bf Hand in:\ }}
\newcommand{\SUBHEAD}[1]{\bigskip\par\noindent\hspace{-.1in}{\sc #1}\\}
%\newenvironment{CHECKLIST}{\begin{itemize}}{\end{itemize}}


\usepackage[compact]{titlesec}

\begin{document}
\MYTITLE{Final Project Assignment: Implementing and Releasing a Twitter Analytics System}
\MYHEADERS{Final Project Assignment}{Due: December 13, 2013 at 5 pm}

\section*{Introduction} 

Throughout this semester, you learned how to use tools that enable you to to specify, design, implement, test, and
document Java programs.  You also had several experiences with working in progressively larger teams to complete the
phases of the software life cycle, with a focus on the elicitation of software requirements, the planning of software
projects, and the analysis of a project's design, implementation, and test suite.  In this final project assignment you
will follow the phases of the software development lifecycle to create and release Twitter analytics software.

\section*{Using the Twitter Service}

Twitter is a microblogging service that allows you to send and receive short, 140-character messages.  While many
Twitter clients allow you to send and receive tweets and search for trending topics, few give you the ability to search
and analyze all of the Tweets that you have sent over the entire time you have used Twitter. If you have tweeted many
times in the past, it is particularly difficult to implement a Twitter client that downloads your entire history of
tweets because of the way in which Twitter rate limits its application programming interface.  Since many people are
interested in search through all of their tweets, Twitter recently announced a mechanism for downloading them all to
your computer, as described at the following Web site \url{https://blog.twitter.com/2012/your-twitter-archive/}.
However, manually searching through this archive is very cumbersome. 

\section*{Requirements for Twitter Analytics}

For this final project, your team will implement a complete Twitter analytics system.  To start, your program should be
able to accept as input the Zip file that a Twitter user downloads by following the instructions in the aforementioned
Web site. Next, your program must parse all of the tweets in the Zip file and store them in a relational database so
that they can be subject to further analysis. Moreover, your analytics tools must allow the user to connect to the
Twitter servers and refresh the program's tweet database so that it includes all of the new tweets since the download of
the Zip file from Twitter. Your system should also allow the user to both specify a new Zip file for reloading into the
local database and reinitialize the database by clearing out any existing tweets.

Beyond being able to store the tweets in a local relational database, your program should furnish several analyses that
the users can perform on their tweets.  For instance, the system should accept a pattern from the user and then return
all of the tweets that match this pattern.  Another feature could involve an analysis of the days of the week or the
times on which the user most frequently tweets.  You should also implement features that identify common words or
phrases in a user's tweets. At minimum, your system should offer at least five useful ways to analyze the full history
of a user's tweets.  Throughout the entire final project, you should interact with your customer to best make
intelligent decisions about what features to include.

At minimum, one version of your system must be implemented in the Java programming language and use JCommander to parse
command-line arguments. Please use SQLite version 3 to store all of the tweets and the information about the tweets.
Your entire system should be tested with JUnit test cases that achieve a high level of coverage according to the JaCoCo
coverage monitor.  You should also regularly analyze the source code and test suite of your system using tools such as
JDepend, JavaNCSS, and MAJOR. Each separate analysis of your program should be accessible through an Ant build system
that also supports compilation, documentation, and cleaning tasks.

In order to implement your Twitter analytics system, you must create a project and register it with the Twitter
developer network.  Please take care to follow all of Twitter's rules in order to ensure that your program is not
prohibited from accessing Twitter. Then, you will need to learn how to write a Java program that can access Twitter
through the Twitter4J system described at \url{http://twitter4j.org/en/index.html}.  See the instructor if you
have questions about Twitter.

\section*{Summary of the Required Deliverables}

This assignment invites your team to submit one printed version of the following files:
\vspace*{-.1in}
\begin{enumerate}
	\itemsep0em 
	\item A description of and justification for your team's chosen organization, roles, and tool support
	\item A description of the internal structure of a downloaded Twitter archive
	\item A description of the format of the file that stores all of the tweets for a single user
	\item Reports from running analysis tools (e.g., JaCoCo, JavaNCSS, JDepend, and MAJOR) 
	\item A complete, fully documented software system that is released on Google Code and includes
	\begin{enumerate}
		\item A distinctive name and logo
		\item Easy-to-understand user documentation for display on a Web site
		\item The source code of your system's program and test suite
		\item A build system that can document, build, test, and analyze your program's source code
	\end{enumerate}

\end{enumerate}
\vspace*{-.1in}

You must also ensure that the instructor has read access to your Bitbucket repository that is named according to the
convention {\tt cs290F2013-fp-team{\em k}}, with {\tt {\em k}} representing the number of your assigned team.  Your
repository should contain all of the deliverables that you produced during the completion of this assignment.  Once your
project is complete, you should place the final version of all public-facing deliverables on a publicly available Google
Code project that uses a Git repository. Please see the instructor if you have any questions about these requirements.

\end{document}
