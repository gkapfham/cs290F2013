\input{labspre.tex}

\usepackage[compact]{titlesec}

\begin{document}
\MYTITLE{Final Project Assignment: Implementing and Releasing a Twitter Analytics System}
\MYHEADERS{Final Project Assignment}{Due: December 13, 2013 at 5 pm}

\section*{Introduction} 

Throughout this semester, you learned how to use tools that enable you to to specify, design, implement, test, and
document Java programs.  You also had several experiences with working in progressively larger teams to complete the
phases of the software life cycle, with a focus on the elicitation of software requirements, the planning of software
projects, and the analysis of a project's design, implementation, and test suite.  In this final project assignment you
will follow the phases of the software development lifecycle to create and release Twitter analytics software.

\section*{Using the Twitter Service}

Twitter is a microblogging service that allows you to send and receive short, 140-character messages.  While many
Twitter clients allow you to send and receive tweets and search for trending topics, few give you the ability to search
and analyze all of the Tweets that you have sent over the entire time you have used Twitter. If you have tweeted many
times in the past, it is particularly difficult to implement a Twitter client that downloads your entire history of
tweets because of the way in which Twitter rate limits its application programming interface.  Since many people are
interested in search through all of their tweets, Twitter recently announced a mechanism for downloading them all to
your computer, as described at the following Web site \url{https://blog.twitter.com/2012/your-twitter-archive/}.
However, manually searching through this archive is very cumbersome. 

\section*{Requirements for Twitter Analytics}

For this final project, your team will implement a complete Twitter analytics system.  To start, your program should be
able to accept as input the Zip file that a Twitter user downloads by following the instructions in the aforementioned
Web site. Next, your program must parse all of the tweets in the Zip file and store them in a relational database so
that they can be subject to further analysis. Moreover, your analytics tools must allow the user to connect to the
Twitter servers and refresh the program's tweet database so that it includes all of the new tweets since the download of
the Zip file from Twitter. Your system should also allow the user to both specify a new Zip file for reloading into the
local database and reinitialize the database by clearing out any existing tweets.

\section*{Summary of the Required Deliverables}

This assignment invites your team to submit one printed version of the following files:
\vspace*{-.1in}
\begin{enumerate}
	\itemsep0em 
	\item A description of and justification for your team's chosen organization, roles, and tool support
	\item A description of the internal structure of a downloaded Twitter archive
	\item A description of the format of the file that stores all of the tweets for a single user
	\item A complete, fully documented software system ready for release on Google Code and including
	\begin{enumerate}
		\item A distinctive name and logo
		\item Easy-to-understand user documentation for display on a Web site
		\item 
	\end{enumerate}

\end{enumerate}
\vspace*{-.1in}

You must also ensure that the instructor has read access to your Bitbucket repository that is named according to the
convention {\tt cs290F2013-fp-team{\em k}}, with {\tt {\em k}} representing the number of your assigned team.  Your
repository should contain all of the deliverables that you produced during the completion of this assignment.  Please
see the instructor if you have any questions. 
 

\end{document}
