\input{labspre.tex}

\usepackage[compact]{titlesec}

\begin{document}
\MYTITLE{Final Project Assignment: Implementing and Releasing a Twitter Analytics System}
\MYHEADERS{Final Project Assignment}{Due: , 2013}

\section*{Introduction} 

Throughout this semester, you learned how to use tools that enable you to to specify, design, implement,
test, and document Java programs.  You also had several experiences with working in progressively larger teams to
complete the phases of the software life cycle, with a recent focus on the elicitation of software requirements, the
planning of software projects, and the analysis of a project's design and implementation.  


In this assignment, you will
install and use a mutation analysis tool that will help you to determine the quality of a test suite.  During the final
project, you can use this tool to better gauge how effectively you are testing your code. A week from now, you will
demonstrate to the instructor that you are able to perform mutation analysis.

\section*{Summary of the Required Deliverables}

This assignment invites your team to submit one printed version of the following files:
\vspace*{-.1in}
\begin{enumerate}
	\itemsep0em 
	\item A description of and justification for your team's chosen organization, roles, and tool support
	\item A description of the targets in build.xml and the order in which they are run during mutation 
	\item A modified version of Triangle.java that contains a {\tt main} method for demonstration purposes
	\item The output from running the modified version of Triangle.java that contains a {\tt main} method
	\item The output from executing the {\tt run.sh} script in the {\tt example/ant} directory
	\item A detailed analysis of the output produced by the execution of the {\tt run.sh} script
	\item A thoughtful commentary on at least one live and at least one killed source code mutant
	\item A listing of the steps that you will take during next week's demonstration of MAJOR
\end{enumerate}
\vspace*{-.1in}

You must also ensure that the instructor has read access to your Bitbucket repository that is named according to the
convention {\tt cs290F2013-lab7-team{\em k}}, with {\tt {\em k}} representing the number of your assigned team.  Your
repository should contain all of the deliverables that you produced during the completion of this assignment.  A team
can earn extra credit if they successfully apply MAJOR to another software system (e.g., a system from the last
laboratory assignment). To earn the extra points you must submit an instruction manual that fully explains how to use
MAJOR to perform mutation analysis on your chosen project. Please see the instructor if you have any questions. 
 

\end{document}
