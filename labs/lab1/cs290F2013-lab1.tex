% CS 580 style
% Typical usage (all UPPERCASE items are optional):
%       \input 580pre
%       \begin{document}
%       \MYTITLE{Title of document, e.g., Lab 1\\Due ...}
%       \MYHEADERS{short title}{other running head, e.g., due date}
%       \PURPOSE{Description of purpose}
%       \SUMMARY{Very short overview of assignment}
%       \DETAILS{Detailed description}
%         \SUBHEAD{if needed} ...
%         \SUBHEAD{if needed} ...
%          ...
%       \HANDIN{What to hand in and how}
%       \begin{checklist}
%       \item ...
%       \end{checklist}
% There is no need to include a "\documentstyle."
% However, there should be an "\end{document}."
%
%===========================================================
\documentclass[11pt,twoside,titlepage]{article}
%%NEED TO ADD epsf!!
\usepackage{threeparttop}
\usepackage{graphicx}
\usepackage{latexsym}
\usepackage{color}
\usepackage{listings}
\usepackage{fancyvrb}
%\usepackage{pgf,pgfarrows,pgfnodes,pgfautomata,pgfheaps,pgfshade}
\usepackage{tikz}
\usepackage[normalem]{ulem}
\tikzset{
    %Define standard arrow tip
%    >=stealth',
    %Define style for boxes
    oval/.style={
           rectangle,
           rounded corners,
           draw=black, very thick,
           text width=6.5em,
           minimum height=2em,
           text centered},
    % Define arrow style
    arr/.style={
           ->,
           thick,
           shorten <=2pt,
           shorten >=2pt,}
}
\usepackage[noend]{algorithmic}
\usepackage[noend]{algorithm}
\newcommand{\bfor}{{\bf for\ }}
\newcommand{\bthen}{{\bf then\ }}
\newcommand{\bwhile}{{\bf while\ }}
\newcommand{\btrue}{{\bf true\ }}
\newcommand{\bfalse}{{\bf false\ }}
\newcommand{\bto}{{\bf to\ }}
\newcommand{\bdo}{{\bf do\ }}
\newcommand{\bif}{{\bf if\ }}
\newcommand{\belse}{{\bf else\ }}
\newcommand{\band}{{\bf and\ }}
\newcommand{\breturn}{{\bf return\ }}
\newcommand{\mod}{{\rm mod}}
\renewcommand{\algorithmiccomment}[1]{$\rhd$ #1}
\newenvironment{checklist}{\par\noindent\hspace{-.25in}{\bf Checklist:}\renewcommand{\labelitemi}{$\Box$}%
\begin{itemize}}{\end{itemize}}
\pagestyle{threepartheadings}
\usepackage{url}
\usepackage{wrapfig}
% removing the standard hyperref to avoid the horrible boxes
%\usepackage{hyperref}
\usepackage[hidelinks]{hyperref}
% added in the dtklogos for the bibtex formatting
\usepackage{dtklogos}
%=========================
% One-inch margins everywhere
%=========================
\setlength{\topmargin}{0in}
\setlength{\textheight}{8.5in}
\setlength{\oddsidemargin}{0in}
\setlength{\evensidemargin}{0in}
\setlength{\textwidth}{6.5in}
%===============================
%===============================
% Macro for document title:
%===============================
\newcommand{\MYTITLE}[1]%
   {\begin{center}
     \begin{center}
     \bf
     CMPSC 290\\Principles of Software Development\\
     Fall 2013
     \medskip
     \end{center}
     \bf
     #1
     \end{center}
}
%================================
% Macro for headings:
%================================
\newcommand{\MYHEADERS}[2]%
   {\lhead{#1}
    \rhead{#2}
    %\immediate\write16{}
    %\immediate\write16{DATE OF HANDOUT?}
    %\read16 to \dateofhandout
    \def \dateofhandout {September 25, 2013}
    \lfoot{\sc Handed out on \dateofhandout}
    %\immediate\write16{}
    %\immediate\write16{HANDOUT NUMBER?}
    %\read16 to\handoutnum
    \def \handoutnum {5}
    \rfoot{Handout \handoutnum}
   }

%================================
% Macro for bold italic:
%================================
\newcommand{\bit}[1]{{\textit{\textbf{#1}}}}

%=========================
% Non-zero paragraph skips.
%=========================
\setlength{\parskip}{1ex}

%=========================
% Create various environments:
%=========================
\newcommand{\PURPOSE}{\par\noindent\hspace{-.25in}{\bf Purpose:\ }}
\newcommand{\SUMMARY}{\par\noindent\hspace{-.25in}{\bf Summary:\ }}
\newcommand{\DETAILS}{\par\noindent\hspace{-.25in}{\bf Details:\ }}
\newcommand{\HANDIN}{\par\noindent\hspace{-.25in}{\bf Hand in:\ }}
\newcommand{\SUBHEAD}[1]{\bigskip\par\noindent\hspace{-.1in}{\sc #1}\\}
%\newenvironment{CHECKLIST}{\begin{itemize}}{\end{itemize}}


\usepackage[compact]{titlesec}

\begin{document}
\MYTITLE{Laboratory Assignment One: Version Control with Git and Bitbucket}
\MYHEADERS{Laboratory Assignment One}{}

\section*{Introduction}

Practicing software engineers normally use a version control system to manage most of the artifacts produced during the
phases of the software development life cycle.  In this course, we will always use the Git distributed version control
system to manage the files associated with our laboratory assignments.  In this laboratory assignment, you will learn
how to use the Bitbucket service for managing Git repositories and the {\tt git} command-line tool in the Ubuntu Linux
operating system.

\section*{Configuring Git and Bitbucket}

During this laboratory assignment and subsequent assignments, we will securely communicate with the Bitbucket.org
servers that will host our all of our projects.  In this laboratory assignment, we will perform all of the steps to
configure the accounts on the departmental servers and the Bitbucket service.  Throughout the assignment, you should
refer to the following Web site for additional information:
\url{https://confluence.atlassian.com/display/BITBUCKET/Bitbucket+101}.  As you will be required to prepare a tutorial
describing each step that you complete in this assignment, please be sure to keep a record of all of the steps that you
complete.  You are also responsible for working with your team members to ensure that every member of the team is able
to successfully complete each of the steps outlined in this assignment.

\begin{enumerate}
	\item If you have never done so before, you must use the {\tt ssh-keygen} program to create secure-shell keys that
		you can use to support your communication with the Bitbucket servers.  Type {\tt man ssh-keygen} and talk with
		the members of your team to learn more about how to use this program.  What files does {\tt ssh-keygen} produce? 
		Where does this program store these files?

	\item If you do not already have a Bitbucket account, please go to the Bitbucket Web site and create one --- 
		make sure that you use your {\tt allegheny.edu} email address so that you can create an unlimited number of free
		Bitbucket repositories.

	\item Now, you need to test to see if you can authenticate with the Bitbucket servers.  Open a terminal window on
		your workstation and change into the directory where you will store your files for this laboratory assignment.
		Then, please type the following command: {\tt git clone git@bitbucket.org:gkapfham/cs290f2013-lab1.git}.  If
		everything worked correctly, you should be able to download all of the files that I used for the introductory
		presentation in Computer Science 290 Fall 2013. Please resolve any problems that your encountered by first
		reviewing the Bitbucket documentation and then working with your team members.

	\item To ensure that you can prepare your own presentation using your own version of the files that you downloaded, please review
		and discuss the files {\tt cs290F2013-introduction.html} and {\tt big.css}. You can learn about this
		presentation framework by visiting: \url{https://github.com/tmcw/big}. Please ask the instructor if you have
		questions about this source code.

\end{enumerate}

\section*{Creating a New Repository}

Now that you have learned how to clone an existing Git repository, you and your team members are responsible for
creating a new repository that contains the source code of a presentation explaining the use of Git. Using screen shots
and easy-to-understand points and the presentation system in the downloaded repository, your presentation should explain
how to use the following Git commands. 

\begin{enumerate} 
			
	\item {\tt git init}

	\item {\tt git status}

	\item {\tt git add} 

	\item {\tt git commit}

	\item {\tt git push}

	\item {\tt git pull} 

	\item One additional {\tt git} command

\end{enumerate}

When you presentation describes a specific Git command, it should explain its input and output with concrete examples.
Your team is responsible for creating one presentation in a manner that ensures each member can make a substantial
contribution. You should use a version control repository to coordinate your work on the presentation.  Students who
would like to learn more about Git can consult Web sites like \url{http://try.github.io/} and
\url{http://gitimmersion.com/}. 

To create a new Git repository that is hosted on the Bitbucket servers, a member of your team should first create a
local directory and then initialize it as a local Git repository.  Next, you should use the Bitbucket
Web site to create a repository that has the same name as the local directory and local repository.  Next, you must
follow Bitbucket's instructions to push the code and tags in your local repository to the one hosted by Bitbucket's
servers.  After completing this step, the chosen team member should share the repository with both the course instructor
and everyone else on the team.  At this point, all members of the team will be able to clone the repository and
manipulate the files stored inside of it.  Now, you must work together to finish the required presentation!

\section*{Summary of the Required Deliverables}

This assignment invites your team to submit one printed version of a tutorial that contains:

\begin{enumerate}
	
	\item A description of the steps that a user must take to configure Git and Bitbucket

	\item A description of the inputs, outputs, and behavior of the aforementioned Git commands

\end{enumerate}

You must also ensure that the instructor has read access to your Bitbucket repository that is named according to the
convention {\tt cs290F2013-lab1-team{\em k}}, with {\tt {\em k}} representing the number of your assigned team. Please
see the instructor if you would like to print your tutorial slides in color.

\end{document}
