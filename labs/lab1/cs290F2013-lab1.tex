\input{labspre.tex}

\usepackage[compact]{titlesec}

\begin{document}
\MYTITLE{Laboratory Assignment One: Version Control with Git and Bitbucket}
\MYHEADERS{Laboratory Assignment One}{}

\section*{Introduction}

Practicing software engineers normally use a version control system to manage most of the artifacts produced during the
phases of the software development life cycle.  In this course, we will always use the Git distributed version control
system to manage the files associated with our laboratory assignments.  In this laboratory assignment, you will learn
how to use the Bitbucket service for managing Git repositories and the {\tt git} command-line tool in the Ubuntu Linux
operating system.

\section*{Configuring Git and Bitbucket}

During this laboratory assignment and subsequent assignments, we will securely communicate with the Bitbucket.org
servers that will host our all of our projects.  In this laboratory assignment, we will perform all of the steps to
configure the accounts on the departmental servers and the Bitbucket service.  Throughout the assignment, you should
refer to the following Web site for additional information:
\url{https://confluence.atlassian.com/display/BITBUCKET/Bitbucket+101}.  As you will be required to prepare a tutorial
describing each step that you complete in this assignment, please be sure to keep a record of all of the steps that you
complete.  You are also responsible for working with your team members to ensure that every member of the team is able
to successfully complete each of the steps outlined in this assignment.

\begin{enumerate}
	\item If you have never done so before, you must use the {\tt ssh-keygen} program to create secure-shell keys that
		you can use to support your communication with the Bitbucket servers.  Type {\tt man ssh-keygen} and talk with
		the members of your team to learn more about how to use this program.  What files does {\tt ssh-keygen} produce? 
		Where does this program store these files?

	\item If you do not already have a Bitbucket account, please go to the Bitbucket Web site and create one.  Please
		make sure that you use your {\tt allegheny.edu} email address so that you can create an unlimited number of free
		repositories.
\end{enumerate}

\section*{Summary of Deliverables}

\end{document}
