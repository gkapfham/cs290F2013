\input{labspre.tex}

\usepackage[compact]{titlesec}

\begin{document}
\MYTITLE{Laboratory Assignment Three: Using Vim to Implement and Test Java Programs}
\MYHEADERS{Laboratory Assignment Three}{Due: September 25, 2013}

\section*{Introduction}

While it is possible to implement and test Java program in Vim without the use of any additional plugins, it is often
useful to extend Vim with a plugin called Eclim.  As described at \url{http://eclim.org/}, Eclim is a plugin that makes
it possible to access the features of Eclipse through Vim.  This means that you can program in Vim and use Eclipse
features such as code completion, searching, validation, and compilation. In this laboratory assignment, you will work
with your team members to learn how to use the basic features associated with Eclim.  Then, leveraging your knowledge of
Git, Vim, and Eclim, you will implement a test suite and user interface for a small Java program.  Finally, you will 
give a demonstration of your working test suite and program.

\section*{Learning the Basics of Eclim}

Eclim provides a convenient and easy-to-use interface to all of the features provided by Eclipse.  Since we will be
using Eclim's Java-based features in Vim, you should study \url{http://eclim.org/vim/java/index.html} to learn how to
use this plugin.  In particular, make sure you can use:

\begin{enumerate}
	\item :PingEclim
	\item :ProjectCreate
	\item :ProjectList
	\item :ProjectInfo
	\item :JavaDocComment
	\item :JavaFormat
	\item Code completion with CTRL-x, CTRL-u 
\end{enumerate}

\section*{Summary of the Required Deliverables}

This assignment invites your team to submit one printed version of the following files:

\begin{enumerate}
	
	\item A description of the input(s), output(s), and behavior of the basic Eclim commands 
	\item 
	\item 

\end{enumerate}

You must also ensure that the instructor has read access to your Bitbucket repository that is named according to the
convention {\tt cs290F2013-lab3-team{\em k}}, with {\tt {\em k}} representing the number of your assigned team. 
Each team should submit both the source code.

\end{document}
