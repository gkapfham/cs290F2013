% CS 580 style
% Typical usage (all UPPERCASE items are optional):
%       \input 580pre
%       \begin{document}
%       \MYTITLE{Title of document, e.g., Lab 1\\Due ...}
%       \MYHEADERS{short title}{other running head, e.g., due date}
%       \PURPOSE{Description of purpose}
%       \SUMMARY{Very short overview of assignment}
%       \DETAILS{Detailed description}
%         \SUBHEAD{if needed} ...
%         \SUBHEAD{if needed} ...
%          ...
%       \HANDIN{What to hand in and how}
%       \begin{checklist}
%       \item ...
%       \end{checklist}
% There is no need to include a "\documentstyle."
% However, there should be an "\end{document}."
%
%===========================================================
\documentclass[11pt,twoside,titlepage]{article}
%%NEED TO ADD epsf!!
\usepackage{threeparttop}
\usepackage{graphicx}
\usepackage{latexsym}
\usepackage{color}
\usepackage{listings}
\usepackage{fancyvrb}
%\usepackage{pgf,pgfarrows,pgfnodes,pgfautomata,pgfheaps,pgfshade}
\usepackage{tikz}
\usepackage[normalem]{ulem}
\tikzset{
    %Define standard arrow tip
%    >=stealth',
    %Define style for boxes
    oval/.style={
           rectangle,
           rounded corners,
           draw=black, very thick,
           text width=6.5em,
           minimum height=2em,
           text centered},
    % Define arrow style
    arr/.style={
           ->,
           thick,
           shorten <=2pt,
           shorten >=2pt,}
}
\usepackage[noend]{algorithmic}
\usepackage[noend]{algorithm}
\newcommand{\bfor}{{\bf for\ }}
\newcommand{\bthen}{{\bf then\ }}
\newcommand{\bwhile}{{\bf while\ }}
\newcommand{\btrue}{{\bf true\ }}
\newcommand{\bfalse}{{\bf false\ }}
\newcommand{\bto}{{\bf to\ }}
\newcommand{\bdo}{{\bf do\ }}
\newcommand{\bif}{{\bf if\ }}
\newcommand{\belse}{{\bf else\ }}
\newcommand{\band}{{\bf and\ }}
\newcommand{\breturn}{{\bf return\ }}
\newcommand{\mod}{{\rm mod}}
\renewcommand{\algorithmiccomment}[1]{$\rhd$ #1}
\newenvironment{checklist}{\par\noindent\hspace{-.25in}{\bf Checklist:}\renewcommand{\labelitemi}{$\Box$}%
\begin{itemize}}{\end{itemize}}
\pagestyle{threepartheadings}
\usepackage{url}
\usepackage{wrapfig}
% removing the standard hyperref to avoid the horrible boxes
%\usepackage{hyperref}
\usepackage[hidelinks]{hyperref}
% added in the dtklogos for the bibtex formatting
\usepackage{dtklogos}
%=========================
% One-inch margins everywhere
%=========================
\setlength{\topmargin}{0in}
\setlength{\textheight}{8.5in}
\setlength{\oddsidemargin}{0in}
\setlength{\evensidemargin}{0in}
\setlength{\textwidth}{6.5in}
%===============================
%===============================
% Macro for document title:
%===============================
\newcommand{\MYTITLE}[1]%
   {\begin{center}
     \begin{center}
     \bf
     CMPSC 290\\Principles of Software Development\\
     Fall 2013
     \medskip
     \end{center}
     \bf
     #1
     \end{center}
}
%================================
% Macro for headings:
%================================
\newcommand{\MYHEADERS}[2]%
   {\lhead{#1}
    \rhead{#2}
    %\immediate\write16{}
    %\immediate\write16{DATE OF HANDOUT?}
    %\read16 to \dateofhandout
    \def \dateofhandout {September 25, 2013}
    \lfoot{\sc Handed out on \dateofhandout}
    %\immediate\write16{}
    %\immediate\write16{HANDOUT NUMBER?}
    %\read16 to\handoutnum
    \def \handoutnum {5}
    \rfoot{Handout \handoutnum}
   }

%================================
% Macro for bold italic:
%================================
\newcommand{\bit}[1]{{\textit{\textbf{#1}}}}

%=========================
% Non-zero paragraph skips.
%=========================
\setlength{\parskip}{1ex}

%=========================
% Create various environments:
%=========================
\newcommand{\PURPOSE}{\par\noindent\hspace{-.25in}{\bf Purpose:\ }}
\newcommand{\SUMMARY}{\par\noindent\hspace{-.25in}{\bf Summary:\ }}
\newcommand{\DETAILS}{\par\noindent\hspace{-.25in}{\bf Details:\ }}
\newcommand{\HANDIN}{\par\noindent\hspace{-.25in}{\bf Hand in:\ }}
\newcommand{\SUBHEAD}[1]{\bigskip\par\noindent\hspace{-.1in}{\sc #1}\\}
%\newenvironment{CHECKLIST}{\begin{itemize}}{\end{itemize}}


\usepackage[compact]{titlesec}

\begin{document}
\MYTITLE{Laboratory Assignment Three: Using Vim to Implement and Test Java Programs}
\MYHEADERS{Laboratory Assignment Three}{Due: September 25, 2013}

\section*{Introduction}

While it is possible to implement and test Java programs in Vim without the use of any additional plugins, it is often
useful to extend Vim with a plugin called Eclim.  As described at \url{http://eclim.org/}, Eclim is a plugin that makes
it possible to access the features of Eclipse through Vim.  This means that you can program in Vim and use Eclipse
features that support the completion, searching, validation, and compilation of source code. In this laboratory
assignment, you will work with your team members to learn how to use the basic features associated with Eclim.  Then,
leveraging your knowledge of Git, Vim, and Eclim, you will implement a test suite and user interface for a Java
program.  In next week's session, you will publicly demonstrate your test suite and program.

\section*{Learning the Basics of Eclim}

Eclim provides a convenient and easy-to-use interface to all of the features provided by Eclipse.  Since we will be
using Eclim's Java-based features in Vim, you should study \url{http://eclim.org/vim/java/index.html} to learn how to
use this plugin.  In particular, make sure you can use:

\begin{enumerate}
	\item {\tt :PingEclim}
	\item {\tt :ProjectCreate}
	\item {\tt :ProjectList}
	\item {\tt :ProjectInfo}
	\item {\tt :JavaDocComment}
	\item {\tt :JavaFormat}
	\item {\tt :JavaImport}
	\item {\tt :JavaSearch}
	\item Code completion with CTRL-x, CTRL-u 
\end{enumerate}

\section*{Implementing and Testing a Java Program}

Once you have finished learning the basics of Eclim, you are ready to access a Bitbucket-hosted Git repository that
contains the source code for the system that you will finish implementing and testing.  First, the members of each team
should ask the instructor to share the Bitbucket repository with them. Once the repository is in your Bitbucket
dashboard, you should use a terminal to execute this command in an appropriate directory: {\tt git clone
git@bitbucket.org:gkapfham/kinetic.git}.

Once you have finishing cloning the repository, please take some time to study the directory structure.  What files did
you find? What did you notice about the directories?  Now, you should examine the .eclimrc\_k file that is responsible
for configuring Eclim to find the Kinetic project that is this assignment's focus.  Please use Vim to change the {\tt
sgi.instance.area.default} variable so that it points to the directory containing the Git repository. 

Now, you are ready to run the Eclim daemon called {\tt eclimd}.  Using the {\tt which} command in your terminal, find
out where this command was previously installed on your workstation.  Once you have verified that {\tt eclimd} is
installed on your machine, run {\tt eclimd} with the {\tt -f} flag to specify that the .eclimrc\_k file should be used
as the runtime configuration file for Eclim.  After running this program, what output do you see in the terminal window?

If {\tt eclimd} is running correctly, then you can type {\tt gvim} in a terminal window from the {\tt kinetic}
directory. Using CTRL-P, load the build.xml file and study the targets that it provides.  Next, use CTRL-P to load the
Kinetic.java, KineticTest.java, and AllTests.java files.  Again, please carefully review these Java classes to make sure
that you understand the provided methods. Once you understand both the build file and the Java classes, you should 
use the {\tt :PingEclim} command to make sure that GVim can connect to the Eclim daemon.  If you see debugging output
showing the current version of both Eclim and Eclipse, then you can issue the {\tt :ProjectCreate} command in GVim to
create an Eclipse project for Kinetic.  Once the project was successfully created, you can run {\tt :ProjectList}
and {\tt :ProjectInfo} to ensure that you properly initialized Eclim. 
 
Since Vim is now configured to operate as an integrated development environment, it contains features that support
the compilation and testing of your Java source code.  For instance, we can use the Apache Ant build system, described
at \url{http://ant.apache.org/}, to perform automated compilation and testing of our program. If you type {\tt :Ant
	compile} you can run the Java compiler for the three Java classes that are part of the project.  Next, call {\tt
:Ant test} to run the test suite.  Finally, do not forget to run the {\tt :Ant clean} command. What output do you see
when you use these commands? Please see the instructor if these commands do not work correctly.

\section*{Performing Test Coverage Monitoring}

Practicing software engineers commonly use test suites to identify defects in and establish a confidence in the
correctness of their programs.  You can run the tests for the Kinetic system by executing the {\tt :Ant test} command.
What output does the test suite produce?  Do the tests find a bug in the Kinetic program?  Of course, it is very
important to evaluate the quality of the test suite that you implement.  Under the assumption that it is not possible to
find bugs in code that has not been executed, software engineers commonly use code coverage to evaluate test quality.  

Once again, we can use Vim and the {\tt :Ant coverage} command to run the JaCoCo coverage monitoring tool.  After
learning more about JaCoCo by visiting \url{http://www.eclemma.org/jacoco/}, you should load the coverage report in the
{\tt bin/site/jacoco/AllTests/edu.allegheny.kinetic/} directory to determine if Kinetic has a good test suite.
How much of the code does {\tt KineticTest} cover?  Is this an acceptable score?  As you continue to enhance Kinetic
and add new features and test cases, you should regularly use JaCoCo to calculate the coverage of your test
suite.  If you notice that you have code or conditional logic branches that are not covered, you should write additional
test cases to cover this code.  Please ensure that you submit both the coverage report that you produced before
modifying Kinetic and the one that you created for the final version.

\section*{Fixing and Extending the Program}

If you carefully studied the Kinetic.java file, you will have noticed that it contains a defect! In light of the fact
that this program has a test suite with tests that always pass, it is alarming to note the existence of this bug. 
Can you write a test case that will reveal this defect? Can you fix the bug?  Finally, you should use Vim and Eclim to
complete the following implementation tasks:

\vspace*{-.1in}
\begin{enumerate}

	\item Add a simple command-line interface that will allow a user to perform velocity computations
	\item Add test cases for any source code associated with the new command-line user interface	
	\item Create a build.xml target that will allow Kinetic to be called with the {\tt :Ant kinetic} command
	\item Add comments to all of the Java classes and methods in Kinetic, KineticTest, and AllTests

\end{enumerate}

\section*{Summary of the Required Deliverables}

This assignment invites your team to submit one printed version of the following files:

\begin{enumerate}
	
	\item A description of the input(s), output(s), and behavior of the basic Eclim commands 
	\item A description of the output(s) and behavior of Vim's implementation and testing commands
	\item The first and final coverage reports produced by the JaCoCo coverage analysis tool
	\item A discussion of the defect that you found in the Kinetic Java class
	\item The final version of the Kinetic, KineticTest, and AllTests Java classes
	\item Screenshots demonstrating that your program and test suite run correctly
\end{enumerate}

You must also ensure that the instructor has read access to your Bitbucket repository that is named according to the
convention {\tt cs290F2013-lab3-team{\em k}}, with {\tt {\em k}} representing the number of your assigned team.  Please
ensure that your team does not modify the source code in the Kinetic repository --- instead, it is your responsibility
to make a new repository that will contain your team's version of Kinetic.  Each team should submit the appropriately
documented and tested source code for Kinetic --- you must turn in all of the Java source code that you created
or modified for this assignment. Please see the instructor if you have questions about these deliverables.

\end{document}
