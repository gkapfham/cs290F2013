\input{labspre.tex}

\usepackage[compact]{titlesec}

\begin{document}
\MYTITLE{Laboratory Assignment Six: Automatically Characterizing Programs and Test Suites}
\MYHEADERS{Laboratory Assignment Six}{Due: November 6, 2013}

\section*{Introduction}

In the previous laboratory assignments, you have learned about the tools that we will use during the remainder of this
semester to specify, design, implement, test, and document Java programs.  You have also had several experiences with
working in progressively larger teams to complete the phases of the software life cycle, with a recent focus on the
elicitation of software requirements and the planning of software projects.  In this assignment, you and your team will
install, configure, and use software tools that automatically calculate design and implementation metrics that
characterize programs and test suites.  One week from now, you will present your findings from your analyses.  

\section*{Calculating Design Quality Metrics with JDepend}

\section*{Observations About the Requirements}

\section*{Summary of the Required Deliverables}

This assignment invites your team to submit one printed version of the following files:
\vspace*{-.1in}
\begin{enumerate}
	\itemsep0em 
	\item A description of and justification for your team's chosen organization, roles, and tool support
	\item A document that clearly specifies the inputs, outputs, and behavior of the next release planner
	\item A document that explains the planner's design, with details about classes and methods
	\item All of the implementation artifacts (e.g., build system, source code, and the coverage report) 
\end{enumerate}
\vspace*{-.1in}

You must also ensure that the instructor has read access to your Bitbucket repository that is named according to the
convention {\tt cs290F2013-lab5-team{\em k}}, with {\tt {\em k}} representing the number of your assigned team.  Your
repository should contain all of the deliverables that you produced during the completion of this assignment.  Please
see the instructor if you have any questions.

\end{document}
