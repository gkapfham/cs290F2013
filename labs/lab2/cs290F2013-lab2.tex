% CS 580 style
% Typical usage (all UPPERCASE items are optional):
%       \input 580pre
%       \begin{document}
%       \MYTITLE{Title of document, e.g., Lab 1\\Due ...}
%       \MYHEADERS{short title}{other running head, e.g., due date}
%       \PURPOSE{Description of purpose}
%       \SUMMARY{Very short overview of assignment}
%       \DETAILS{Detailed description}
%         \SUBHEAD{if needed} ...
%         \SUBHEAD{if needed} ...
%          ...
%       \HANDIN{What to hand in and how}
%       \begin{checklist}
%       \item ...
%       \end{checklist}
% There is no need to include a "\documentstyle."
% However, there should be an "\end{document}."
%
%===========================================================
\documentclass[11pt,twoside,titlepage]{article}
%%NEED TO ADD epsf!!
\usepackage{threeparttop}
\usepackage{graphicx}
\usepackage{latexsym}
\usepackage{color}
\usepackage{listings}
\usepackage{fancyvrb}
%\usepackage{pgf,pgfarrows,pgfnodes,pgfautomata,pgfheaps,pgfshade}
\usepackage{tikz}
\usepackage[normalem]{ulem}
\tikzset{
    %Define standard arrow tip
%    >=stealth',
    %Define style for boxes
    oval/.style={
           rectangle,
           rounded corners,
           draw=black, very thick,
           text width=6.5em,
           minimum height=2em,
           text centered},
    % Define arrow style
    arr/.style={
           ->,
           thick,
           shorten <=2pt,
           shorten >=2pt,}
}
\usepackage[noend]{algorithmic}
\usepackage[noend]{algorithm}
\newcommand{\bfor}{{\bf for\ }}
\newcommand{\bthen}{{\bf then\ }}
\newcommand{\bwhile}{{\bf while\ }}
\newcommand{\btrue}{{\bf true\ }}
\newcommand{\bfalse}{{\bf false\ }}
\newcommand{\bto}{{\bf to\ }}
\newcommand{\bdo}{{\bf do\ }}
\newcommand{\bif}{{\bf if\ }}
\newcommand{\belse}{{\bf else\ }}
\newcommand{\band}{{\bf and\ }}
\newcommand{\breturn}{{\bf return\ }}
\newcommand{\mod}{{\rm mod}}
\renewcommand{\algorithmiccomment}[1]{$\rhd$ #1}
\newenvironment{checklist}{\par\noindent\hspace{-.25in}{\bf Checklist:}\renewcommand{\labelitemi}{$\Box$}%
\begin{itemize}}{\end{itemize}}
\pagestyle{threepartheadings}
\usepackage{url}
\usepackage{wrapfig}
% removing the standard hyperref to avoid the horrible boxes
%\usepackage{hyperref}
\usepackage[hidelinks]{hyperref}
% added in the dtklogos for the bibtex formatting
\usepackage{dtklogos}
%=========================
% One-inch margins everywhere
%=========================
\setlength{\topmargin}{0in}
\setlength{\textheight}{8.5in}
\setlength{\oddsidemargin}{0in}
\setlength{\evensidemargin}{0in}
\setlength{\textwidth}{6.5in}
%===============================
%===============================
% Macro for document title:
%===============================
\newcommand{\MYTITLE}[1]%
   {\begin{center}
     \begin{center}
     \bf
     CMPSC 290\\Principles of Software Development\\
     Fall 2013
     \medskip
     \end{center}
     \bf
     #1
     \end{center}
}
%================================
% Macro for headings:
%================================
\newcommand{\MYHEADERS}[2]%
   {\lhead{#1}
    \rhead{#2}
    %\immediate\write16{}
    %\immediate\write16{DATE OF HANDOUT?}
    %\read16 to \dateofhandout
    \def \dateofhandout {September 25, 2013}
    \lfoot{\sc Handed out on \dateofhandout}
    %\immediate\write16{}
    %\immediate\write16{HANDOUT NUMBER?}
    %\read16 to\handoutnum
    \def \handoutnum {5}
    \rfoot{Handout \handoutnum}
   }

%================================
% Macro for bold italic:
%================================
\newcommand{\bit}[1]{{\textit{\textbf{#1}}}}

%=========================
% Non-zero paragraph skips.
%=========================
\setlength{\parskip}{1ex}

%=========================
% Create various environments:
%=========================
\newcommand{\PURPOSE}{\par\noindent\hspace{-.25in}{\bf Purpose:\ }}
\newcommand{\SUMMARY}{\par\noindent\hspace{-.25in}{\bf Summary:\ }}
\newcommand{\DETAILS}{\par\noindent\hspace{-.25in}{\bf Details:\ }}
\newcommand{\HANDIN}{\par\noindent\hspace{-.25in}{\bf Hand in:\ }}
\newcommand{\SUBHEAD}[1]{\bigskip\par\noindent\hspace{-.1in}{\sc #1}\\}
%\newenvironment{CHECKLIST}{\begin{itemize}}{\end{itemize}}


\usepackage[compact]{titlesec}

\begin{document}
\MYTITLE{Laboratory Assignment Two: Using Vim as an Integrated Development Environment}
\MYHEADERS{Laboratory Assignment Two}{Due: September 18, 2013}

\section*{Introduction}

Practicing software engineers normally use an integrated development environment (IDE) to manage various tasks
associated with the design, implementation, and testing of software. In this course, we will use the Vim as an IDE.  In
this laboratory assignment, you will work with your team members to learn about the basic features associated with Vim
and prepare a tutorial that explains how to use Vim plugins to manage tasks in the software development lifecycle.

\section*{Learning the Basics of Vim}

Before you start to use Vim during this laboratory assignment, you may want to review some of the reason why people like
to use this text editor, as explained at \url{http://usevim.com/2012/10/26/why-vim/}.  When you are finished learning
about some of the reasons behind using Vim, you can start to use the Vim text editor in a terminal window or the GVim
text editor in a stand-alone window.  Please work with your team members to identify, learn, and document some of the
basic features that are offered by Vim.  For instance, make sure that you know how to perform the following actions.
Students are encouraged to learn how to use Vim with key commands.

\begin{enumerate}

	\item Open, close, and save files in windows or tabs

	\item Move to the beginning and end of a file

	\item Navigate to specific lines and columns within a file

	\item Enter and exit command mode

	\item Enter and exit insert and append mode

	\item Select line(s) of text in visual mode

	\item Copy, paste, and delete lines of text

	\item Undoing the result of a previous command

	\item Search for and replace specific words in a file 

	\item Additional features that your team deems to be useful

\end{enumerate}

Since we will be using Vim throughout the semester, please make sure that you can easily use all of the editor's most
important commands.  You should take notes and screenshots to demonstrate that your team understands how to use basic
Vim commands.  Then, your team members must work together to create a simple presentation that explains these commands. 

\section*{Basic Configuration of Vim}

It is very easy to configure Vim by writing VimScript in your {\tt .vimrc} and {\tt .gvimrc} files.  To complete the
next phase of the assignment you should download the file
\url{http://www.cs.allegheny.edu/~gkapfham/teach/cs290f2013/labs/lab2/provide/vim-cs290F2013.tar.gz}.  After saving this
file to the root of your home directory, you should decompress it using the {\tt tar} command in your terminal window.
Now, please restart Vim.  Do you see that the color scheme is different? If you would like, you can customize the color
scheme by using the ``Edit'' and ``Color Scheme'' menus. 

What other features have now been added to Vim?  To learn more about how I have configured Vim for use in Computer
Science 290 Fall 2013, you should study the VimScript in the {\tt .vimrc} and {\tt .gvimrc} files.  Make sure that you
and your team members understand these configuration files.

\section*{Using Plugins to Extend Vim}


\section*{Summary of the Required Deliverables}

This assignment invites your team to submit one printed version of a tutorial that contains:

\begin{enumerate}
	
	\item A description of the basic features associated with the Vim text editor

\end{enumerate}

You must also ensure that the instructor has read access to your Bitbucket repository that is named according to the
convention {\tt cs290F2013-lab2-team{\em k}}, with {\tt {\em k}} representing the number of your assigned team. Please
see the instructor if you would like to print your tutorial slides in color.

\end{document}
