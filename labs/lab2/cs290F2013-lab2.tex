\input{labspre.tex}

\usepackage[compact]{titlesec}

\begin{document}
\MYTITLE{Laboratory Assignment Two: Using Vim as an Integrated Development Environment}
\MYHEADERS{Laboratory Assignment Two}{Due: September 18, 2013}

\section*{Introduction}

Practicing software engineers normally use an integrated development environment (IDE) to manage various tasks
associated with the design, implementation, and testing of software. In this course, we will use the Vim as an IDE.  In
this laboratory assignment, you will work with your team members to learn about the basic features associated with Vim
and prepare a tutorial that explains how to use Vim plugins to manage tasks in the software development lifecycle.

\section*{Learning the Basics of Vim}

Before you start to use Vim during this laboratory assignment, you may want to review some of the reason why people like
to use this text editor, as explained at \url{http://usevim.com/2012/10/26/why-vim/}.  When you are finished learning
about some of the reasons behind using Vim, you can start to use the Vim text editor in a terminal window or the GVim
text editor in a stand-alone window.  Please work with your team members to identify, learn, and document some of the
basic features that are offered by Vim.  For instance, make sure that you know how to perform the following actions.
Students are encouraged to learn how to use Vim with key commands.

\begin{enumerate}

	\item Open, close, and save files in windows or tabs

	\item Navigate to specific lines and columns within a file

	\item Enter and exit command mode

	\item Enter and exit insert and append mode

	\item Select line(s) of text in visual mode

	\item Copy, paste, and delete lines of text

	\item Search for specific words in a file 

	\item Additional features that your team deems to be appropriate

\end{enumerate}

\section*{Summary of the Required Deliverables}

This assignment invites your team to submit one printed version of a tutorial that contains:

\begin{enumerate}
	
	\item A description of the basic features associated with the Vim text editor

\end{enumerate}

You must also ensure that the instructor has read access to your Bitbucket repository that is named according to the
convention {\tt cs290F2013-lab2-team{\em k}}, with {\tt {\em k}} representing the number of your assigned team. Please
see the instructor if you would like to print your tutorial slides in color.

\end{document}
